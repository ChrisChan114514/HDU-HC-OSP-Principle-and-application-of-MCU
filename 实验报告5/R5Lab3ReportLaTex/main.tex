\documentclass[12pt,hyperref,a4paper,UTF8]{ctexart}
\usepackage{HDUReport}
\usepackage{listings}
\usepackage{xcolor}
\usepackage{graphicx}
\usepackage{setspace}
\usepackage{float}
\setstretch{1.5} % 设置全局行距为1.5倍

\usepackage{enumitem} % 载入enumitem包以便自定义列表环境
\setlist[itemize]{itemsep=0pt, parsep=0pt} % 设置itemize环境的项目间距和段落间距

\setmainfont{Times New Roman} % 英文正文为Times New Roman


\usepackage{tikz}
\usetikzlibrary{shapes.geometric, arrows}
\usetikzlibrary{positioning, arrows.meta}
\usetikzlibrary{calc}
%封面页设置
{   
    %标题
    \title{ 
        \vspace{1cm}
        \heiti \Huge \textbf{《单片机原理及应用》作业报告} \par
        \vspace{1cm} 
        \heiti \Large {\underline{实验报告5第三部分:自定义波形生成}   } 
        \vspace{3cm}
    
    }

    \author{
        \vspace{0.5cm}
        \kaishu\Large 学院\ \dlmu[9cm]{卓越学院} \\ %学院
        \vspace{0.5cm}
        \kaishu\Large 学号\ \dlmu[9cm]{23040447} \\ %班级
        \vspace{0.5cm}
        \kaishu\Large 姓名\ \dlmu[9cm]{陈文轩} \qquad  \\ %学号
        \vspace{0.5cm}
        \kaishu\Large 专业\ \dlmu[9cm]{智能硬件与系统(电子信息工程)} \qquad \\ %姓名 
    }
        
    \date{\today} % 默认为今天的日期,可以注释掉不显示日期
}
%%------------------------document环境开始------------------------%%
\begin{document}

%%-----------------------封面--------------------%%
\cover
\thispagestyle{empty} % 首页不显示页码
%%------------------摘要-------------%%
%\newpage
%\begin{abstract}




%\end{abstract}

%\thispagestyle{empty} % 首页不显示页码

%%--------------------------目录页------------------------%%
% \newpage
% \tableofcontents
% \thispagestyle{empty} % 目录不显示页码

%%------------------------正文页从这里开始-------------------%
\newpage
\setcounter{page}{1} % 让页码从正文开始编号

%%可选择这里也放一个标题
%\begin{center}
%    \title{ \Huge \textbf{{标题}}}
%\end{center}


\textbf{原题目:DACO832与单片机的接线如课堂上所接,参考电压为-2V,请编程实现波形从+1V开始增大到+2V,再下降到0v,不断循环,周期为20+作业号,单位是ms。}


\section{实验代码}

\begin{lstlisting}[language=C, caption={实验程序}]

#include <reg51.h>
#define DATA P2
#define STUDENT_ID 47   // 学号定义为47

// 波形输出总时间 = (学号 + 20)ms
#define TOTAL_TIME (STUDENT_ID/1.375 + 20)  // 单位:ms
unsigned char mode = 0;  // 0:自定义波形, 1:锯齿波, 2:三角波

// 自定义波形表 - 从+1V到+2V再到0V
// 0V对应值为0,1V对应值为128,2V对应值为255
unsigned char code customWave[256] = {
    // 0-15: 起始点(+1V)到上升过程
    128, 132, 136, 140, 144, 148, 152, 156, 160, 164, 168, 172, 176, 180, 184, 188,
    // 16-31: 继续上升
    192, 196, 200, 204, 208, 212, 216, 220, 224, 228, 232, 236, 240, 244, 248, 252,
    // 32-47: 到达峰值(+2V)并开始下降
    255, 255, 250, 245, 240, 235, 230, 225, 220, 215, 210, 205, 200, 195, 190, 185,
    // 48-63: 继续下降
    180, 175, 170, 165, 160, 155, 150, 145, 140, 135, 130, 125, 120, 115, 110, 105,
    // 64-79: 继续下降
    100, 95, 90, 85, 80, 75, 70, 65, 60, 55, 50, 45, 40, 35, 30, 25,
    // 80-95: 接近0V
    20, 15, 10, 5, 0, 0, 0, 0, 0, 0, 0, 0, 0, 0, 0, 0,
    // 96-111: 保持0V一段时间
    0, 0, 0, 0, 0, 0, 0, 0, 0, 0, 0, 0, 0, 0, 0, 0,
    // 112-127: 开始回升
    5, 10, 15, 20, 25, 30, 35, 40, 45, 50, 55, 60, 65, 70, 75, 80,
    // 128-143: 继续回升
    85, 90, 95, 100, 105, 110, 115, 120, 125, 130, 135, 140, 145, 150, 155, 160,
    // 144-159: 继续回升至起始点(+1V)
    165, 170, 175, 180, 185, 190, 195, 200, 205, 210, 215, 220, 225, 230, 235, 240,
    // 160-255: 保持1V电平直到循环结束
    128, 128, 128, 128, 128, 128, 128, 128, 128, 128, 128, 128, 128, 128, 128, 128,
    128, 128, 128, 128, 128, 128, 128, 128, 128, 128, 128, 128, 128, 128, 128, 128,
    128, 128, 128, 128, 128, 128, 128, 128, 128, 128, 128, 128, 128, 128, 128, 128,
    128, 128, 128, 128, 128, 128, 128, 128, 128, 128, 128, 128, 128, 128, 128, 128,
    128, 128, 128, 128, 128, 128, 128, 128, 128, 128, 128, 128, 128, 128, 128, 128,
    128, 128, 128, 128, 128, 128, 128, 128, 128, 128, 128, 128, 128, 128, 128, 128
};


unsigned int timer_count = 0;  // 定时器计数

// 计算锯齿波数据
unsigned char getSawWave(unsigned char i)
{
    // 锯齿波是线性上升的波形
    return i;  // 值从0线性增加到255
}

// 计算三角波数据
unsigned char getTriangleWave(unsigned char i)
{
    if(i < 128)
        return i << 1;  // 0-255上升段
    else
        return 255 - ((i - 128) << 1);  // 255-0下降段
}

// 定时器0初始化
void Timer0Init()
{
    TMOD = 0x01;    // 设置定时器0为模式1(16位定时器)
    EA = 1;         // 开总中断
    ET0 = 1;        // 开定时器0中断
    TR0 = 1;        // 启动定时器0
}

// 设置定时器初值,使每个点的延时为 TOTAL_TIME/256 ms
void setTimer0()
{
    unsigned int delay_us = (TOTAL_TIME * 1000UL) / 256;  // 每个点的延时(微秒)
    unsigned int reload_value;
    
    // 假设使用12MHz晶振,每个机器周期为1us
    reload_value = 65536 - delay_us;
    TH0 = (unsigned char)(reload_value >> 8);
    TL0 = (unsigned char)reload_value;
}

// 定时器0中断服务函数
void Timer0_ISR() interrupt 1
{
    timer_count = 1;  // 设置标志,表示延时完成
    setTimer0();      // 重新设置定时器初值
}
 
void main()
{
    unsigned char i;
    unsigned char waveData;
    
    Timer0Init();   // 初始化定时器
    setTimer0();    // 设置定时器初值
    
    while(1)
    {
        for(i=0;i<256;i++)
        {
            // 根据mode选择波形
            switch(mode)
            {
                case 0:  // 自定义波形(使用预先计算好的波形表)
                    waveData = customWave[i];
                    break;
                case 1:  // 锯齿波
                    waveData = getSawWave(i);
                    break;
                case 2:  // 三角波
                    waveData = getTriangleWave(i);
                    break;
                default:
                    waveData = customWave[i];  // 默认使用自定义波形
                    break;
            }
            DATA = waveData;
            
            // 等待定时器中断发生(延时结束)
            timer_count = 0;
            while(!timer_count); 
        }
    }
}
\end{lstlisting}

\section{实验效果}

\begin{figure}[H] % [H] 表示强制当前位置插入
    \centering
    \includegraphics[width=0.9\textwidth]{figures/201.png} % 调整宽度为文本宽度的 80%
    \caption{代码控制效果} %图片标题
    \label{fig:example} % 图片标签,用于引用
\end{figure}
示波器光标测量得:周期为67ms,符合要求;DC挡位测量电压变化为0至2V,符合实验预期。
\begin{figure}[H] % [H] 表示强制当前位置插入
    \centering
    \includegraphics[width=0.9\textwidth]{figures/202.png} % 调整宽度为文本宽度的 80%
    \caption{电路结构} %图片标题
    \label{fig:example} % 图片标签,用于引用
\end{figure}

\section{流程图}


\begin{figure}[H] % [H] 表示强制当前位置插入
        \centering
        \includegraphics[width=0.9\textwidth]{figures/301.png} % 调整宽度为文本宽度的 80%
        \caption{系统控制流程图} %图片标题
        \label{fig:example} % 图片标签,用于引用
\end{figure}



\section{实验体会}

通过本次自定义波形生成实验,我获得了以下几点体会与收获:

\begin{enumerate}
    \item \textbf{DAC原理与应用的初步理解}:本实验使用DAC0832数模转换器与单片机配合工作,加深了我对DAC工作原理的理解。通过实践发现,数字量到模拟量的转换过程需要精确的时序控制和值映射,这对应用中实现准确的电压输出至关重要。
    
    \item \textbf{波形设计与实现}:通过查表法实现波形输出是一种高效的方法。在设计自定义波形表时,需要精确计算电压与数字量之间的对应关系,例如将0-2V的电压范围映射到0-255的数字量范围。这种映射关系的掌握对于后续更复杂波形的生成有重要意义。
    
    \item \textbf{定时控制的重要性}:波形生成中,时间控制直接影响波形的周期特性。本实验要求实现(20+学号)ms的周期波形,这需要精确计算每个点的输出时间。通过定时器中断方式实现精确延时,比简单的软件延时更为可靠。
    
    \item \textbf{硬件连接的规范性}:DAC0832与单片机的接线需要严格遵循规范,任何连接错误都会导致输出波形异常。这让我认识到在实际工程应用中,硬件接口的正确连接是系统稳定运行的基础。
\end{enumerate}

总的来说,这次实验将理论知识与实际操作紧密结合,帮助我建立了从数字系统到模拟世界的桥梁。通过自己动手设计并实现特定要求的波形,不仅加深了对单片机与外设协同工作的理解,也提高了解决实际问题的能力。

\end{document}